\documentclass[11pt]{article}
\usepackage[utf8]{inputenc}
\usepackage[letterpaper,top=0cm, margin=0.85in]{geometry}

%for math
\usepackage{amsmath, amssymb, amsfonts} %standard
\usepackage{youngtab} % makes squares for math diagrams
%-----------------------------------------------------------           

%\usepackage{sectsty}
%for lists and numbers
\usepackage{enumitem}
%-----------------------------------------------------------

% Doc setting
\usepackage[english]{babel} % Replace `english' with e.g. `spanish' to change the document language
\usepackage{textcmds} %more symbols
\usepackage{fontspec} %more fonts
\usepackage{setspace} %to set spacing bw words and lines
\usepackage{changepage}
\setlength\parindent{0pt}

%footer
\usepackage{fancyhdr}
\usepackage{lastpage}

\fancyhf{} % sets both header and footer to nothing
\renewcommand{\headrulewidth}{0pt} %remove headerline

\fancyfoot[RE,RO]{\thepage}
\fancyfoot[LE,LO]{\emph{Shubhro Gupta — CS-2210}}
\pagestyle{fancy}
%-----------------------------------------------------------

%for pictures and graphs
\usepackage{graphicx} %add image
\usepackage{adjustbox}

\usepackage{pgfplots} %for graphing plotting
\pgfplotsset{compat=1.18, width=10cm}
%-----------------------------------------------------------

%for code
\usepackage{verbatim}
\usepackage{listings}
\usepackage{fancyvrb} %for coding blocks
%\usepackage{algorithm}
%\usepackage{algpseudocode} %for pseudocode
%\usepackage{algorithm, algpseudocode}
\usepackage[linesnumbered]{algorithm2e}
\usepackage{algorithm2e}

\setmonofont[Scale=MatchLowercase]{[SFMono-Regular.ttf]}
%\usepackage{lstfiracode} %firacode
\usepackage[framemethod=tikz]{mdframed} %adding background to lstlisting
\usepackage{tikz-3dplot} 

\usepackage[ruled,vlined,boxed]{algorithm2e} %for pseudocode lines



%for colors and links
\usepackage[colorlinks = true,
            linkcolor = blue,
            urlcolor  = blue,
            citecolor = blue,
            anchorcolor = blue]{hyperref}
\usepackage[many]{tcolorbox}  % for colored boxes
\usepackage{color} % to get colors
\usepackage{xcolor} %more colors options and flexibility
\usepackage{transparent}



%-----------------------------------------------------------------------------


%CUSOMIZATIONS

% my colors
%dracula
\definecolor{background}{rgb}{0.16,0.16,0.21}
\definecolor{codegreen}{rgb}{0.24,0.68,0.65}
\definecolor{codepurple}{rgb}{0.51,0.31,0.87}
\definecolor{codered}{rgb}{0.81,0.13,0.18}
\definecolor{codebluegray}{rgb}{0.02,0.31,0.68}

\definecolor{comment}{rgb}{0.67,0.74,0.79}
\definecolor{textcolor}{rgb}{0.22,0.22,0.22}

%style for coding
\lstdefinestyle{python}{
    language=python,
    backgroundcolor=\color{white},   
    commentstyle={\color{comment}},
    keywordstyle={\color{codepurple}},
    stringstyle={\color{codegreen}},
    basicstyle={\ttfamily\color{textcolor}},
    keywordstyle = [2]{\color{codered}},
    keywordstyle = [3]{\color{codebluegray}},
    keywordstyle = [4]{\color{teal}},
    otherkeywords = {<, >, +, -, =, *, \[, \], &&, ||, format, 1, 2, 3, 4, 5, 6, 7, 8, 9, 0, ;},
    morekeywords = [4]{+, -, *, /, =, <, >, format},
    morekeywords = [3]{\[, \],  1, 2, 3, 4, 5, 6, 7, 8, 9, 0, ;},
    %
    breakatwhitespace=false, 
    frame=shadowbox,
    rulecolor=\color{textcolor},
    breaklines=true,                 
    captionpos=b,                    
    keepspaces=true,                 
    numbers=left,
    numbersep=15pt, %distance between code and numbers
    numberstyle=\scriptsize\ttfamily\color{comment},
    showspaces=false,                
    showstringspaces=false,
    showtabs=false,
    xleftmargin=4.3em, %margin bw left page and frame
    framexleftmargin=3.8em, %margin bw text and frame
    %xleftmargin=3.4em,
    framexrightmargin=-0.5em,
    tabsize=2,
    aboveskip=1.5em,
    belowskip=0.5em,
    framextopmargin=9pt,
    framexbottommargin=9pt,
    frameshape={RYR}{Y}{Y}{RYR}
}

\lstdefinestyle{c}{
    language=c,
    backgroundcolor=\color{white},   
    commentstyle={\color{comment}},
    keywordstyle={\color{codepurple}},
    stringstyle={\color{codegreen}},
    basicstyle={\ttfamily\color{textcolor}},
    keywordstyle = [2]{\color{codered}},
    keywordstyle = [3]{\color{codebluegray}},
    keywordstyle = [4]{\color{teal}},
    otherkeywords = {<, >, +, -, =, *, \[, \], &&, ||, stdio.h, stdlib.h, 1, 2, 3, 4, 5, 6,7, 8, 9, 0, ;},
    morekeywords = [4]{+, -, *, /, =, <, >, stdio.h, stdlib.h},
    morekeywords = [3]{\[, \],  1, 2, 3, 4, 5, 6, 7, 8, 9, 0, ;},
    morekeywords = [2]{&&, ||},
    %
    breakatwhitespace=false, 
    frame=shadowbox,
    rulecolor=\color{textcolor},
    breaklines=true,                 
    captionpos=b,                    
    keepspaces=true,                 
    numbers=left,                    
    numbersep=15pt, %distance between code and numbers
    numberstyle=\scriptsize\ttfamily\color{comment},
    showspaces=false,                
    showstringspaces=false,
    showtabs=false,
    xleftmargin=4.3em, %margin bw left page and frame
    framexleftmargin=3.8em, %margin bw text and frame
    %xleftmargin=3.4em,
    framexrightmargin=-0.5em,
    tabsize=2,
    aboveskip=1.5em,
    belowskip=0.5em,
    framextopmargin=9pt,
    framexbottommargin=9pt,
    frameshape={RYR}{Y}{Y}{RYR}
}

%-----------------------------------------------------------------------------
%custom commands

%code
\newcommand{\problem}[1]{\begin{adjustwidth}{0.1px}\noindent \framebox[1.2\width]{\large Problem #1}\end{adjustwidth} \bigskip\\}
\newcommand{\codecap}[2]{{\vspace{4px}{\emph{#1}}} \hfill \href{#2}{Link to the code\ }\vspace{25px}}
\newcommand{\code}[1]{{\texttt{#1}}}

\SetKwInput{KwInput}{Input}
\SetKwInput{KwOutput}{Output}

%math
\newcommand{\bigo}[1]{$O(#1)$ }
\newcommand{\thetan}[1]{$\theta(#1)$}
\newcommand{\vector}[1]{$\overrightarrow{#1}$}

%display
\newcommand{\link}[3][blue]{\href{#2}{\color{#1}{#3}}}%
\newcommand{\inlink}[1]{\underline{\emph{\link[black]{#1}{#1}}}}


%header
\newcommand{\lesgo}[5]{
\begin{large}
\emph{#1}\smallskip \\
\textbf{Shubhro Gupta} \hfill Week #2\smallskip \\
Professor #3 \hfill Due #4\\
\end{large} \medskip \\
{\emph{Collaborators: #5}}\\
\hrule
\vspace{50px}
\\
}

\newcommand\dunderline[3][-1pt]{{%
  \sbox0{#3}%
  \ooalign{\copy0\cr\rule[\dimexpr#1-#2\relax]{\wd0}{#2}}}}

%new section
\newcommand{\asec}[1]{{\vspace{20px}\large\dunderline[-3px]{1px}{\textbf{#1}}} \\}




%-----------------------------------------------------------------------------
%title
\usepackage{algpseudocode}
\begin{document}

\lesgo{CS-2210 Linear Algebra}{1}{Pritam Ghosh}{September 26, 2022}{none}

\problem{1.b, 1.e}
\textbf{ATQ}
\begin{itemize}
    \item Let $O$ denote the origin and $P, Q$ denote some other point in $\mathbb{R}^3$.
    \item Let $O & = (0, 0, 0), P = (0, 1, 1),$ and $Q = (2, 0, 1)$.
\end{itemize}
\\
\textbf{To Draw / Prove}
\begin{enumerate}[label=(\alph*)]
    \item [(b)] Draw a vector diagram to establish the relation between $\overrightarrow{OP}, \overrightarrow{OQ},$ and $\overrightarrow{PQ}$.
    \item [(e)] Is the set of vectors $\{\overrightarrow{OP}, \overrightarrow{OQ}, \overrightarrow{PQ}\}$ linearly independent?
\end{enumerate} \bigskip
\\
\textbf{Figure / Proof}
    
\begin{enumerate}[label=(\alph*)]
    \item [(b)]
   $\overrightarrow{OP} = (0, 1, 1), \ \overrightarrow{PQ} = (2, -1, 0), \ \overrightarrow{OQ} = (2, 0, 1)$. 
\begin{figure}[htp]
\begin{center}
\tdplotsetmaincoords{60}{120} 
\begin{tikzpicture} [scale=1.4, tdplot_main_coords, axis/.style={->,color=grey,thick}, 
vector/.style={-stealth,very thick}, 
vector guide/.style={dashed,color=comment,thick}]

%standard tikz coordinate definition using x, y, z coords
\coordinate (O) at (0,0,0);

%tikz-3dplot coordinate definition using x, y, z coords

\pgfmathsetmacro{\ax}{3}
\pgfmathsetmacro{\ay}{3}
\pgfmathsetmacro{\az}{3}

\coordinate (O) at (0,0,0);
\coordinate (P) at (0, 1, 1);
\coordinate (Q) at (2, 0, 1);

%draw guide lines to components
\draw[vector guide] (\ax,0,\az) -- (\ax, 0, 0);
\draw[vector guide] (2,0,\az) -- (2, 0, 0);
\draw[vector guide] (1,0,\az) -- (1, 0, 0);

\draw[vector guide] (\ax,0,\az) -- (0, 0, \az);
\draw[vector guide] (\ax,0,2) -- (0, 0, 2);
\draw[vector guide] (\ax,0,1) -- (0, 0, 1);

\draw[vector guide] (\ax,\ay,0) -- (0,\ay,0);
\draw[vector guide] (\ax,2,0) -- (0,2,0);
\draw[vector guide] (\ax,1,0) -- (0,1,0);

\draw[vector guide] (\ax,\ay,0) -- (\ax,0,0);
\draw[vector guide] (2,\ay,0) -- (2,0,0);
\draw[vector guide] (1,\ay,0) -- (1,0,0);

\draw[vector guide] (0,\ay,\az) -- (0,\ay,0);
\draw[vector guide] (0,2,\az) -- (0,2,0);
\draw[vector guide] (0,1,\az) -- (0,1,0);

\draw[vector guide] (0,\ay,\az) -- (0,0,\az);
\draw[vector guide] (0,\ay,2) -- (0,0,2);
\draw[vector guide] (0,\ay,1) -- (0,0,1);

%draw axes
\draw[axis] (0,0,0) -- (3,0,0) node[anchor=north east]{$x$};
\draw[axis] (0,0,0) -- (0,3,0) node[anchor=south west]{$y$};
\draw[axis] (0,0,0) -- (0,0,3) node[anchor=south]{$z$};

%draw a vector from O to P
\draw[vector, color=blue] (O) -- (P) node[anchor=west]{$\overrightarrow{OP}$};
\draw[vector, color=codepurple] (O) -- (Q) node[anchor=north]{$\overrightarrow{OQ}$};
\draw[vector, color=codered] (P) -- (Q) node[anchor=south]{$\overrightarrow{PQ}$};

\node[tdplot_main_coords,anchor=east]
at (\ax,0,0){(\ax, 0, 0)};
\node[tdplot_main_coords,anchor=west]
at (0,\ay,0){(0, \ay, 0)};
\node[tdplot_main_coords,anchor=east]
at (0,0,\az){(0, 0, \az)};
\end{tikzpicture}
\end{center}
\caption{Relation between $\overrightarrow{OP}, \overrightarrow{OQ} \text{ and } \overrightarrow{PQ}$} \label{Fig 1.}
\end{figure}\bigskip
\\Observing Figure 1, we can notice that $\boldsymbol{\overrightarrow{OP} + \overrightarrow{PQ} = \overrightarrow{OQ}}$.
\bigskip
\\ We can verify this mathematically too, since $
\begin{bmatrix}
0 \\ 1 \\ 1
\end{bmatrix} 
+
\begin{bmatrix}
2 \\ -1 \\ 1
\end{bmatrix} 
= 
\begin{bmatrix}
2 \\ 0 \\ 1
\end{bmatrix}.
$
    
    \item [(e)]
    \textbf{To Determine} \medskip \\
Whether the given set $\{\overrightarrow{OP}, \overrightarrow{OQ} , \overrightarrow{PQ}\} = \left\{
  \begin{bmatrix}
   0 \\ 1 \\ 1
   \end{bmatrix}
   ,\begin{bmatrix}
   2 \\ -1 \\ 1
   \end{bmatrix}
   ,
   \begin{bmatrix}
   2 \\ 0 \\ 1
   \end{bmatrix} \right\}$ is linearly independent.
\bigskip
\\
\textbf{Proof} 
\\
The given set is clearly not independent. \\ In the Figure 1, $\overrightarrow{OP} + \overrightarrow{PQ} = \overrightarrow{OQ}$, thereby proving that the vector $\overrightarrow{OQ}$ can be depicted by the other two vectors.\medskip
\\
Hence the set $\{\overrightarrow{OP}, \overrightarrow{OQ}, \overrightarrow{PQ}\}$ is \textbf{linearly dependent} and spans $\mathbb{R}^2$.
   
   \end{enumerate}
   
   \vspace{3cm}
   \begin{center}
       \emph{Question 2 in the next page}
       
   \end{center}


\newpage
\problem{2}

\begin{tikzpicture}[scale=0.7]
\draw (0,0) node[anchor=north]{$A$}
  -- (4,0) node[anchor=north]{$C$}
  -- (4,4) node[anchor=south]{$B$}
  -- cycle;
  
\draw[dashed] (2.25,0) node[anchor=north]{$N$}
-- (2.25, 2.25) node[anchor=south]{$M$}
\end{tikzpicture}
\hskip 50px\parbox[b]{8cm}{
  \textbf{ATQ}
\\
$M &$ = midpoint of side $AB$\\
$N &$ = midpoint of side $AC$
\bigskip
\\
\textbf{To Prove} \medskip \\
$\overrightarrow{MN}$= $\frac{1}{2}$ $\overrightarrow{BC}$, i.e. $\overrightarrow{MN}$ is parallel to $\overrightarrow{BC}$ and $\frac{1}{2}$ its length. \bigskip
\\
}
\bigskip
\\
\textbf{Proof}
\begin{center}
\begin{tikzpicture}[scale=1]
\draw (0,0) node[anchor=north]{$A$}
  -- (4,0) node[anchor=north]{$C$}
  -- (4,4) node[anchor=south]{$B$}
  -- cycle;
  
\draw[dashed] (2.25,0) node[anchor=north]{$N$}
-- (2.25, 2.25) node[anchor=south]{$M$}\\

\draw[dashed] (0, 3.6) node[anchor=south]{$O$}
-- (4, 4) node[anchor=south]{}\\
\draw[dashed] (0, 3.6) node[anchor=north]{}
-- (2.25, 2.25) node[anchor=south]{}\\
\draw[dashed] (0, 3.6) node[anchor=north]{}
-- (0, 0) node[anchor=south]{}\\
\draw[dashed] (0, 3.6) node[anchor=north]{}
-- (4, 0) node[anchor=south]{}\\
\draw[dashed] (0, 3.6) node[anchor=north]{}
-- (2.25, 0) node[anchor=south]{}\\
\end{tikzpicture}
    \end{center}
\\
\begin{align}
AN & =NC \notag \\
AM & = MB \notag \\
    \overrightarrow{ON} + \overrightarrow{NC} & = \overrightarrow{OC} \notag \\
    \overrightarrow{ON} & = \overrightarrow{OA} + \overrightarrow{NC}\notag \\
    \Rightarrow \overrightarrow{NC} & = \overrightarrow{ON} + \overrightarrow{OA}\notag\\
    \overrightarrow{ON} + \overrightarrow{ON} - \overrightarrow{OA} & = \overrightarrow{OC}\notag\\
    \Rightarrow \overrightarrow{ON} & = \frac{\overrightarrow{OA} + \overrightarrow{OC}}{2}\\
    \Rightarrow \overrightarrow{OM} & = \frac{\overrightarrow{OA} + \overrightarrow{OB}}{2}
\end{align}
\begin{center}
    Putting the value of $\overrightarrow{OA}$ from (2) in (1) we get the following equation
\end{center}
\begin{align*}
    2(\overrightarrow{ON}) - \overrightarrow{OC} & = 2(\overrightarrow{OM}) - \overrightarrow{OB}\\
    2(\overrightarrow{ON} - \overrightarrow{OM}) & = \overrightarrow{OC} - \overrightarrow{OB}\\
    2\overrightarrow{MN} & = \overrightarrow{BC}\\
    \overrightarrow{MN} & = \frac{1}{2}\times \overrightarrow{BC}
\end{align*}
Since \textbf{\boldmath{$MN$ is $\frac{1}{2}\times BC$}}, they are linearly dependent meaning they also have the same direction. Therefore $MN$ and $BC$ are \textbf{parallel}. 

\end{document}
