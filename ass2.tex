\documentclass[11pt]{article}
\usepackage[utf8]{inputenc}
\usepackage[letterpaper,top=0cm, margin=0.85in]{geometry}

%for math
\usepackage{amsmath, amssymb, amsfonts} %standard
\usepackage{youngtab} % makes squares for math diagrams
%-----------------------------------------------------------           

%\usepackage{sectsty}
%for lists and numbers
\usepackage{enumitem}
%-----------------------------------------------------------

% Doc setting
\usepackage[english]{babel} % Replace `english' with e.g. `spanish' to change the document language
\usepackage{textcmds} %more symbols
\usepackage{fontspec} %more fonts
\usepackage{setspace} %to set spacing bw words and lines
\usepackage{changepage}
\setlength\parindent{0pt}

%footer
\usepackage{fancyhdr}
\usepackage{lastpage}

\fancyhf{} % sets both header and footer to nothing
\renewcommand{\headrulewidth}{0pt} %remove headerline

\fancyfoot[RE,RO]{\thepage}
\fancyfoot[LE,LO]{\emph{Shubhro Gupta — CS-2210}}
\pagestyle{fancy}
%-----------------------------------------------------------

%for pictures and graphs
\usepackage{graphicx} %add image
\usepackage{adjustbox}

\usepackage{pgfplots} %for graphing plotting
\pgfplotsset{compat=1.18, width=10cm}
%-----------------------------------------------------------

%for code
\usepackage{verbatim}
\usepackage{listings}
\usepackage{fancyvrb} %for coding blocks
%\usepackage{algorithm}
%\usepackage{algpseudocode} %for pseudocode
%\usepackage{algorithm, algpseudocode}
\usepackage[linesnumbered]{algorithm2e}
\usepackage{algorithm2e}

\setmonofont[Scale=MatchLowercase]{[SFMono-Regular.ttf]}
%\usepackage{lstfiracode} %firacode
\usepackage[framemethod=tikz]{mdframed} %adding background to lstlisting
\usepackage[ruled,vlined,boxed]{algorithm2e} %for pseudocode lines



%for colors and links
\usepackage[colorlinks = true,
            linkcolor = blue,
            urlcolor  = blue,
            citecolor = blue,
            anchorcolor = blue]{hyperref}
\usepackage[many]{tcolorbox}  % for colored boxes
\usepackage{color} % to get colors
\usepackage{xcolor} %more colors options and flexibility
\usepackage{transparent}



%-----------------------------------------------------------------------------


%CUSOMIZATIONS

% my colors
%dracula
\definecolor{background}{rgb}{0.16,0.16,0.21}
\definecolor{codegreen}{rgb}{0.24,0.68,0.65}
\definecolor{codepurple}{rgb}{0.51,0.31,0.87}
\definecolor{codered}{rgb}{0.81,0.13,0.18}
\definecolor{codebluegray}{rgb}{0.02,0.31,0.68}

\definecolor{comment}{rgb}{0.67,0.74,0.79}
\definecolor{textcolor}{rgb}{0.22,0.22,0.22}

%style for coding
\lstdefinestyle{python}{
    language=python,
    backgroundcolor=\color{white},   
    commentstyle={\color{comment}},
    keywordstyle={\color{codepurple}},
    stringstyle={\color{codegreen}},
    basicstyle={\ttfamily\color{textcolor}},
    keywordstyle = [2]{\color{codered}},
    keywordstyle = [3]{\color{codebluegray}},
    keywordstyle = [4]{\color{teal}},
    otherkeywords = {<, >, +, -, =, *, \[, \], &&, ||, format, 1, 2, 3, 4, 5, 6, 7, 8, 9, 0, ;},
    morekeywords = [4]{+, -, *, /, =, <, >, format},
    morekeywords = [3]{\[, \],  1, 2, 3, 4, 5, 6, 7, 8, 9, 0, ;},
    %
    breakatwhitespace=false, 
    frame=shadowbox,
    rulecolor=\color{textcolor},
    breaklines=true,                 
    captionpos=b,                    
    keepspaces=true,                 
    numbers=left,
    numbersep=15pt, %distance between code and numbers
    numberstyle=\scriptsize\ttfamily\color{comment},
    showspaces=false,                
    showstringspaces=false,
    showtabs=false,
    xleftmargin=4.3em, %margin bw left page and frame
    framexleftmargin=3.8em, %margin bw text and frame
    %xleftmargin=3.4em,
    framexrightmargin=-0.5em,
    tabsize=2,
    aboveskip=1.5em,
    belowskip=0.5em,
    framextopmargin=9pt,
    framexbottommargin=9pt,
    frameshape={RYR}{Y}{Y}{RYR}
}

\lstdefinestyle{c}{
    language=c,
    backgroundcolor=\color{white},   
    commentstyle={\color{comment}},
    keywordstyle={\color{codepurple}},
    stringstyle={\color{codegreen}},
    basicstyle={\ttfamily\color{textcolor}},
    keywordstyle = [2]{\color{codered}},
    keywordstyle = [3]{\color{codebluegray}},
    keywordstyle = [4]{\color{teal}},
    otherkeywords = {<, >, +, -, =, *, \[, \], &&, ||, stdio.h, stdlib.h, 1, 2, 3, 4, 5, 6,7, 8, 9, 0, ;},
    morekeywords = [4]{+, -, *, /, =, <, >, stdio.h, stdlib.h},
    morekeywords = [3]{\[, \],  1, 2, 3, 4, 5, 6, 7, 8, 9, 0, ;},
    morekeywords = [2]{&&, ||},
    %
    breakatwhitespace=false, 
    frame=shadowbox,
    rulecolor=\color{textcolor},
    breaklines=true,                 
    captionpos=b,                    
    keepspaces=true,                 
    numbers=left,                    
    numbersep=15pt, %distance between code and numbers
    numberstyle=\scriptsize\ttfamily\color{comment},
    showspaces=false,                
    showstringspaces=false,
    showtabs=false,
    xleftmargin=4.3em, %margin bw left page and frame
    framexleftmargin=3.8em, %margin bw text and frame
    %xleftmargin=3.4em,
    framexrightmargin=-0.5em,
    tabsize=2,
    aboveskip=1.5em,
    belowskip=0.5em,
    framextopmargin=9pt,
    framexbottommargin=9pt,
    frameshape={RYR}{Y}{Y}{RYR}
}

%-----------------------------------------------------------------------------
%custom commands

%code
\newcommand{\problem}[1]{\begin{adjustwidth}{0.1px}\noindent \framebox[1.2\width]{\large Problem #1}\end{adjustwidth} \bigskip\\}
\newcommand{\codecap}[2]{{\vspace{4px}{\emph{#1}}} \hfill \href{#2}{Link to the code\ }\vspace{25px}}
\newcommand{\code}[1]{{\texttt{#1}}}

\SetKwInput{KwInput}{Input}
\SetKwInput{KwOutput}{Output}

%math
\newcommand{\bigo}[1]{$O(#1)$ }
\newcommand{\thetan}[1]{$\theta(#1)$}
\newcommand{\vector}[1]{$\overrightarrow{#1}$}

%display
\newcommand{\link}[3][blue]{\href{#2}{\color{#1}{#3}}}%
\newcommand{\inlink}[1]{\underline{\emph{\link[black]{#1}{#1}}}}


%header
\newcommand{\lesgo}[5]{
\begin{large}
\emph{#1}\smallskip \\
\textbf{Shubhro Gupta} \hfill Week #2\smallskip \\
Professor #3 \hfill Due #4\\
\end{large} \medskip \\
{\emph{Collaborators: #5}}\\
\hrule
\vspace{50px}
\\
}

\newcommand\dunderline[3][-1pt]{{%
  \sbox0{#3}%
  \ooalign{\copy0\cr\rule[\dimexpr#1-#2\relax]{\wd0}{#2}}}}

%new section
\newcommand{\asec}[1]{{\vspace{20px}\large\dunderline[-3px]{1px}{\textbf{#1}}} \\}




%-----------------------------------------------------------------------------
%title
\usepackage{algpseudocode}
\begin{document}

\lesgo{CS-2210 Linear Algebra}{2}{Pritam Ghosh}{September 26, 2022}{none}

\problem{2.a}
\textbf{ATQ}\medskip
\\
$\vec{b} = \begin{bmatrix}
        2 \\
        \llap{-}1 \\
        1
    \end{bmatrix}
$
, set 
$
A = \left\{ \begin{bmatrix}
        1 \\
        5 \\
        2
    \end{bmatrix}, \begin{bmatrix}
        6 \\
        8 \\
        10
    \end{bmatrix}, \begin{bmatrix}
        2\\
        \llap{-}1\\
        3
    \end{bmatrix} 
    \right\}
$
\bigskip
\\
\textbf{To Check}\\
Whether $\vec{b} \in$  span$(A)$.
\bigskip
\\
\textbf{Proof}\\
Let the vectors in set $A$ be $\vec{v_1}, \vec{v_2}, \vec{v_3}$.\\
Let the scalars be denoted as $x_n$.\\
We need to find scalar $x_n$ to be equal to $\vec{b}$ to find whether $\vec{b}$ is in the span($E$).\\
$x_1\vec{v_1} + x_2\vec{v_2} + x_3\vec{v_3} = \vec{b}$
\
\begin{center}
    \text{Convert to augmented matrix}
\end{center}
\vspace*{-\baselineskip}
\begin{align*}
    \begin{bmatrix}
    1 & 6 & 2\\
    5 & 8 & \llap{-}1\\
    2 & 10 & 3
    \end{bmatrix}
    \begin{bmatrix}
    x_1 \\
    x_2 \\
    x_3
    \end{bmatrix}
    =
    \begin{bmatrix}
    2 \\
    \llap{-}1 \\
    1
    \end{bmatrix}
\end{align*}
\begin{center}
    \text{Find the pivot in the 1st column in the 1st row
}
\vspace*{-\baselineskip}
\end{center}
\begin{align*}
    \begin{bmatrix}
    1 & 6 & 2 &\bigm|  2\\
    5 & 8 & \llap{-}1 &\bigm| \llap{-}1\\
    2 & 10 & 3 &\bigm| 1
    \end{bmatrix}
\end{align*}
\begin{center}
    \text{Eliminate the 1st column}
\end{center}
\vspace*{-\baselineskip}
\begin{align*}
    \begin{bmatrix}
    1 & 6 & 2 &\bigm| \  2\\
    0 &  \llap{-}22 & \llap{-}11 &\bigm| \ \llap{-1}1\\
    0 & \llap{-}2 & \llap{-}1 &\bigm| \ \llap{-}3\\
    \end{bmatrix}
\end{align*}
\begin{center}
    Make the pivot in the 2nd column by dividing the 2nd row by -22
\end{center}
\vspace*{-\baselineskip}
\begin{align*}
    \begin{bmatrix}
    1 & 6 & 2 &\bigm| \ 2\\
    0 & 1 & 0.5 &\bigm| \ \llap{0.}5\\
     0 & \llap{-}2 & \llap{-}1&\bigm| \ \llap{-}3
    \end{bmatrix}\\
\end{align*}
\newpage
\begin{center}
    Eliminate the 2nd column
\end{center}
\vspace*{-\baselineskip}
\begin{align*}
    \begin{bmatrix}
    1 & 0 & \llap{-}1 &\bigm| \ \llap{-}1\\
    0 & 1 & 0.5 &\bigm| \ \llap{0.}5\\
     0 & 0 &0 &\bigm| \ \llap{-}1
    \end{bmatrix}\\
\end{align*}
As we can see, the system is inconsistent and has \textbf{no solutions}. Therefore it \textbf{does not span $A$.}
\vspace{3cm}
   \begin{center}
       \emph{Question 6.b in the next page}
       
   \end{center}
\newpage

\problem{6.b}
\textbf{To Determine} \medskip \\
Whether the given set $\left\{
  \begin{bmatrix}
   3 \\ 1 \\ 1
   \end{bmatrix}
   ,\begin{bmatrix}
   0 \\ 1 \\ 0
   \end{bmatrix}
   ,
   \begin{bmatrix}
   5 \\ 0 \\ 4
   \end{bmatrix} \right\}$ is linearly independent.
\bigskip
\\
\textbf{Proof} 
\\
The matrix \begin{bmatrix}
   3 & 0 & 5 & \bigm|0\\
   1 & 1 & 0 & \bigm|0\\
   1 & 0 & 4 & \bigm|0
   \end{bmatrix}, would be independent if the vectors reduce to \begin{bmatrix}
   1 & 0 & 0 & \bigm|0\\
   0 & 1 & 0 & \bigm|0\\
   0 & 0 & 1 & \bigm|0
   \end{bmatrix}.\\
   
The only solution to the equation $c_1\vec{v_1}+ c_2\vec{v_2}+ c_3\vec{v_3} = 0$ must be $c_1, c_2 = c_3 = 0$ for it to be linearly independent.
\begin{center}
Our matrix
\end{center}
\vspace*{-\baselineskip}
\begin{align*}
    \begin{bmatrix}
    3 & 0 & 5 &\bigm|  0\\
    1 & 1 & 0 &\bigm| 0\\
    1 & 0 & 4 &\bigm| 0
    \end{bmatrix}
\end{align*}
\begin{center}
First and second row = Swap the second and the first rows
\end{center}
\vspace*{-\baselineskip}
\begin{align*}
    \begin{bmatrix}
    1 & 1 & 0 &\bigm| 0\\
    3 & 0 & 5 &\bigm|  0\\
    1 & 0 & 4 &\bigm| 0
    \end{bmatrix}
\end{align*}
\begin{center}
First row = first row $\times$ 3
\end{center}
\vspace*{-\baselineskip}
\begin{align*}
    \begin{bmatrix}
    3 & 3 & 0 &\bigm| 0\\
    3 & 0 & 5 &\bigm|  0\\
    1 & 0 & 4 &\bigm| 0
    \end{bmatrix}
\end{align*}
\begin{center}
Second row = second row -  first row\\
First row = restored
\end{center}
\vspace*{-\baselineskip}
\begin{align*}
    \begin{bmatrix}
    1 & 1 & 0 &\bigm| 0\\
    0 & \llap{-}3 & 5 &\bigm|  0\\
    1 & 0 & 4 &\bigm| 0
    \end{bmatrix}
\end{align*}
\begin{center}
Third row = third row - first row
\end{center}
\vspace*{-\baselineskip}
\begin{align*}
    \begin{bmatrix}
    1 & 1 & 0 &\bigm| 0\\
    0 & \llap{-}3 & 5 &\bigm|  0\\
    0 & \llap{-}1 & 4 &\bigm| 0
    \end{bmatrix}
\end{align*}
\begin{center}
Swap the third row and second row\\
Inverse the sign in the second row
\end{center}
\vspace*{-\baselineskip}
\begin{align*}
    \begin{bmatrix}
    1 & 1 & 0 &\bigm| 0\\
    0 & 1 & \llap{-}4 &\bigm| 0\\
    0 & \llap{-}3 & 5 &\bigm|  0\\
    \end{bmatrix}
\end{align*}
\begin{center}
First row = first row = second row
\end{center}
\vspace*{-\baselineskip}
\begin{align*}
    \begin{bmatrix}
    1 & 0 & 4 &\bigm| 0\\
    0 & 1 & \llap{-}4 &\bigm| 0\\
    0 & \llap{-}3 & 5 &\bigm|  0\\
    \end{bmatrix}
\end{align*}
\begin{center}
Second row = second row $\times$ -3
\end{center}
\vspace*{-\baselineskip}
\begin{align*}
    \begin{bmatrix}
    1 & 0 & 4 &\bigm| 0\\
    0 & \llap{-}3 & 12 &\bigm| 0\\
    0 & \llap{-}3 & 5 &\bigm|  0\\
    \end{bmatrix}
\end{align*}
\begin{center}
Third row = third row - second row
Restore the second row
\end{center}
\vspace*{-\baselineskip}
\begin{align*}
    \begin{bmatrix}
    1 & 0 & 4 &\bigm| 0\\
    0 & 1 & \llap{-}4 &\bigm| 0\\
    0 & 0 & \llap{-}7 &\bigm| 0\\
    \end{bmatrix}
\end{align*}
\begin{center}
Third row = $\frac{\text{third row}}{-\text{7}}$
\end{center}
\vspace*{-\baselineskip}
\begin{align*}
    \begin{bmatrix}
    1 & 0 & 4 &\bigm| 0\\
    0 & 1 & \llap{-}4 &\bigm| 0\\
    0 & 0 & 1 &\bigm| 0\\
    \end{bmatrix}
\end{align*}
\begin{center}
First row = first row - third row
\end{center}
\vspace*{-\baselineskip}
\begin{align*}
    \begin{bmatrix}
    1 & 0 & 0 &\bigm| 0\\
    0 & 1 & \llap{-}4 &\bigm| 0\\
    0 & 0 & 4 &\bigm| 0\\
    \end{bmatrix}
\end{align*}
\begin{center}
Third row = third row $\times -1$ 
\end{center}
\vspace*{-\baselineskip}
\begin{align*}
    \begin{bmatrix}
    1 & 0 & 0 &\bigm| 0\\
    0 & 1 & \llap{-}4 &\bigm| 0\\
    0 & 0 & \llap{-}4 &\bigm| 0\\
    \end{bmatrix}
\end{align*}
\begin{center}
Second row = second row - third row\\
Third row = restored
\end{center}
\vspace*{-\baselineskip}
\begin{align*}
    \begin{bmatrix}
    1 & 0 & 0 &\bigm| 0\\
    0 & 1 & 0 &\bigm| 0\\
    0 & 0 & 1 &\bigm| 0\\
    \end{bmatrix}
\end{align*}
Since we can reduce the original matrix to it's \underline{reduced echelon form}, hence the vectors are \textbf{linearly independent} and would span $\mathbb{R}^3$.
 
\end{document}
